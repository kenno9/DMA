\documentclass[12pt,a4paper]{article}

% Packages
\usepackage[english,danish]{babel}
\usepackage[applemac]{inputenc}
\usepackage{amsmath,amscd}
\usepackage{amssymb}
\usepackage{amsthm}
\usepackage{enumerate}
\usepackage{graphicx}                    
\usepackage{framed}  
\usepackage{xcolor}

\theoremstyle{plain}
\newtheorem{thm}{S�tning}
\newtheorem{prop}[thm]{Proposition}
\newtheorem{lem}[thm]{Lemma}
\newtheorem{cor}[thm]{Korollar}
\newtheorem{conj}[thm]{Formodning}
\theoremstyle{definition}
\newtheorem{exercise}{Opgave}
\newtheorem{definition}[thm]{Definition}
\newtheorem{prob}[thm]{Problem}
\newtheorem{remark}[thm]{Bem�rkning}
\newtheorem{example}[thm]{Eksempel}


% Blackboard bold
\newcommand{\NN}{\mathbb{N}}
\newcommand{\ZZ}{\mathbb{Z}}
\newcommand{\QQ}{\mathbb{Q}}
\newcommand{\RR}{\mathbb{R}}
\newcommand{\CC}{\mathbb{C}}
\newcommand{\GCD}{\operatorname{GCD}}



\begin{document}
\begin{center}
\textbf{\large DMA 2018 \\[0.5em] --Ugeopgave 5 -- \\[0.5em]}
\end{center}

\begin{itemize}
\item Hele ugeopgaven skal besvares.
\item Ugeopgaven skal afleveres mandag den 8. oktober klokken 21:59 p� Absalon.
\item Ugeopgaven skal laves i \textbf{grupper} af 3-4 personer.
\item Besvarelsen skal udarbejdes i \LaTeX.
\end{itemize}


\begin{enumerate}[Del 1]
\item N�r vi benytter Euklids algoritme p� to tal $a,b$ for at bestemme $\GCD(a,b)$ foretager vi et antal divisioner med rest indtil vi opn�r resten $0$ og dermed har bestemt den st�rste f�lles divisor som den n�stsidst beregnede rest. Vi vil sige at antallet af \textbf{trin} der skal benyttes er antallet af divisioner. S�ledes er antallet af trin der skal benyttes for at bestemme $\GCD(273,98)$ netop 5, jf.\ gennemregningen i KBR Example 1.4.5 (side 23). Antallet af trin for alle valg af $a,b$ med $15\geq a\geq b>0$ -- p� n�r to s�danne valg -- er illustreret i figur \ref{antaltrin}.
\begin{enumerate}[(1)]
\item Beregn $\GCD(3,2), \GCD (5,3), \GCD (8,5)$, samt $\GCD(13,8)$ og
  bestem de fire manglende tal i figur \ref{antaltrin}.

\item Lad $t_n$ v�re det h�jeste (worst-case) antal trin der skal benyttes til at bestemme $\GCD(a,b)$ n�r $n\geq a\geq b>0$. Benyt figur \ref{antaltrin} til at bestemme $t_1,t_2,\dots,t_{15}$.
\item For $k=2,3,4,5,6$, find par $(a_k,b_k)$ s�ledes hver
  $\GCD(a_k,b_k)$ har netop $k$ divisioner, og $\max\{a_k,b_k\}$
  bliver mindst muligt. \textit{Hint: Du kan med fordel benytte tabel
    i Figur~\ref{antaltrin}}. Du kan antage at $(a_6,b_6)=(21,13)$.
\item \textit{(Frivillig -- man beh�ves ikke at lave denne opgave)}
  Kan du indse m�nstret og forudsige $(a_7,b_7)$ og $(a_8,b_8)$?
Derefter, overvej f�lgen defineret som $F_0=1$, $F_1=2$, og
$F_k=\max\{a_k,b_k\}$ for $k>1$. Kan du genkende f�lgen $(F_k)$ fra
forel�sningen? Hvad hedder den?
\item Vis at $t_n$ er $O(n)$. \textit{Note: grafen for $t_n$ for $n$
    mellem 1 og 200 er illustreret p� figur \ref{vaekst}. Bem�rk, ud
    fra grafen s� ser $t_n$ ud ikke som $\Theta(n)$.}
\item Giv en begrundelse for at $t_n$ ikke er $O(1)$.
\end{enumerate}
%%%%%%%%%%%%%%%%%
%%%%%%%%%%%%%%%%%
%%%%%%%%%%%%%%%%%
\item
 Benyt  f�lgende opskrift
  til at give et induktionsbevis for at  $4^{n} + 15n - 1$ er deleligt med $9$ for ethvert helt tal $n>0$.
\begin{enumerate}[(1)]\item Bestem det relevante udsagn $P(n)$.
\item Kontroll�r at $P(n)$ er et sandt udsagn for alle $n$ mellem $1$
og $5$.
\item Indf�r en f�lge $b_n = 4^{n} + 15n  -1$, og lav en formel der
  sammenknytter $b_{n+1}$ og $b_{n}$
\item Antag nu at $P(n)$ er sand for en eller anden bestemt v�rdi af $n>0$. G�r rede for at s� er $P(n+1)$ ogs� sand.
\item Opstil en  konklusion ved hj�lp af induktionsprincippet.
\end{enumerate}
\end{enumerate}
\newpage



\begin{figure}
\begin{center}
\begin{tabular}{|c||c|c|c|c|c|c|c|c|c|c|c|c|c|c|c|}\hline
&1&2&3&4&5&6&7&8&9&10&11&12&13&14&15\\\hline\hline
1&1&1&1&1&1&1&1&1&1&1&1&1&1&1&1\\\hline
2&&1&&1&2&1&2&1&2&1&2&1&2&1&2\\\hline
3&&&1&2&&1&2&3&1&2&3&1&2&3&1\\\hline
4&&&&1&2&2&3&1&2&2&3&1&2&2&3\\\hline
5&&&&&1&2&3&&3&1&2&3&4&3&1\\\hline
6&&&&&&1&2&2&2&3&3&1&2&2&2\\\hline
7&&&&&&&1&2&3&3&4&4&3&1&2\\\hline
8&&&&&&&&1&2&2&4&2& &3&3\\\hline
9&&&&&&&&&1&2&3&2&3&4&3\\\hline
10&&&&&&&&&&1&2&2&3&3&2\\\hline
11&&&&&&&&&&&1&2&3&4&4\\\hline
12&&&&&&&&&&&&1&2&2&2\\\hline
13&&&&&&&&&&&&&1&2&3\\\hline
14&&&&&&&&&&&&&&1&2\\\hline
15&&&&&&&&&&&&&&&1\\\hline
\end{tabular}
\end{center}
\caption{Antal trin i beregningen af $\GCD(a,b)$}\label{antaltrin}
\end{figure}


\begin{figure}
\begin{center}
\includegraphics[width=0.45\textwidth, trim=2.5cm 12cm 5cm 2cm, clip=true]{vaekst}
\end{center}
\caption{Grafen for $t_n$} \label{vaekst}
\end{figure}
\end{document}
